\documentclass[answers]{exam}
\usepackage{amsmath}
\usepackage{amssymb}
\usepackage{amsthm}
\usepackage{color}
\usepackage[utf8]{inputenc}
\footer{}{\thepage}{}

\SolutionEmphasis{\color{red}}

\title{Solutions to Weekly Questions}
\author{Mathematics Society}
\date{\texttt{3-11-2019}}

\begin{document}
%This code creates the text before the first question
%-------------------------------------------------------------------


\maketitle


\textbf{Remarks:}
\begin{itemize}
    \item Note that the marks allocated \textbf{is directly proportional} to the difficulty.
    \item Bonus questions are \textbf{not necessarily more difficult}.
    \item The solution will be released on the following Wednesday.
\end{itemize}

\vspace{\stretch{.2}}




%-------------------------------------------------------------------


\begin{questions}

\question Given \(ax^2 + bx + c = 0\) and \(b^2 > 4ac\), derive the quadratic equation \(x=\frac{-b\pm\sqrt{b^{2}-4ac}}{2a}\).


\begin{parts}

\part[2 \half] Find \(e, f, g\) expressed by \(a, b, c\) such that \(ax^2 + bx + c = e(x+f)^2 + g\). \\
Hint: \( g < 0 \).


\begin{solution}
\begin{align*}
ax^2+bx+c &= 0 \\
ax^2+bx+c &= 0 \\
x^2+\frac{b}{a}x+\frac{c}{a} &= 0 \\
x^2+2x\left(\frac{b}{2a}\right)+\left(\frac{b}{2a}\right)^2-\left(\frac{b}{2a}\right)^2+\frac{c}{a} &= 0 \\
\left(x+\frac{b}{2a}\right)^2-\frac{b^2}{4a^2}+\frac{c}{a} &= 0 \\
\left(x+\frac{b}{2a}\right)^2+\frac{b^2-4ac}{4a^2} &= 0 \\
\end{align*}

$$e=1,\, f=\frac{b}{2a},\, g=\frac{b^2-4ac}{4a^2}$$

\end{solution}

\vspace{\stretch{1}}

\newpage

\part[\half] Using (a), derive the quadratic equation.

\begin{solution}
\begin{align*}
\left(x+\frac{b}{2a}\right)^2+\frac{b^2-4ac}{4a^2} &= 0 \\
\left(x+\frac{b}{2a}\right)^2 &= \frac{b^2-4ac}{4a^2} \\
x+\frac{b}{2a} &= \pm\sqrt{\frac{b^2-4ac}{4a^2}} \\
x+\frac{b}{2a} &= \frac{\pm\sqrt{b^2-4ac}}{2a} \\
x &= \frac{-b\pm\sqrt{b^2-4ac}}{2a}
\end{align*}
\end{solution}

\vspace{\stretch{1}} 

\end{parts}

\newpage



\question Given two coordinates \( (a,b)\) and \((c, d)\).

\begin{parts}

\part[1 \half] Show that if \(ad - bc \neq 0\), then there \textbf{does not exist} \(e,f \neq 0  \) such that \( (ea, eb) + (fc, fd) = (0, 0) \). \\ 

\begin{solution}
\begin{proof}

Assume there exists such $e,f$ pair. Then,

\begin{align*}
ea+fc &= 0 \\
ea &= -fc\\ 
a &= -\frac{f}{e}c\\
\end{align*}

\begin{align*}
eb+fd &= 0 \\
eb &= -fd\\ 
b &= -\frac{f}{e}d\\
\end{align*}

As such,
\begin{align*}
ad-bc &= -\frac{f}{e}cd +\frac{f}{e}dc \\
&= 0
\end{align*}
Which contradicts the fact that \( ad-bd \neq 0 \). Hence such pairs do not exist.

\end{proof}

\end{solution}

\newpage

\part[1 \half]  Show that if \(ad - bc = 0\), then there \textbf{exists} \(e,f \neq 0  \) such that \( (ea, eb) + (fc,fd) = (0, 0) \). \\ 

\begin{solution}
\begin{proof}
\begin{align*}
ad-bc &= 0 \\
ad &= bc
\end{align*}
If $a$ or $d$ $= 0$ then either $b$ or $c$ $= 0$. It could be easily seen that the solution is trivial. As such, let us assume \(a,b,c,d \neq 0\)

\begin{align*}
ea+fc &= 0 \\
ea &= -fc\\ 
\frac{a}{c} &= -\frac{f}{e}
\end{align*}

\begin{align*}
eb+fd &= 0 \\
eb &= -fd\\ 
\frac{b}{d} &= -\frac{f}{e}
\end{align*}

$ e \neq 0 \because \frac{a}{c} $ is defined.  \(f \neq 0 \because \frac{a}{c} \neq 0\) As such, there exists $e,f\neq 0$ that satisfy the equation.

\end{proof}
\end{solution}


\vspace{\stretch{1}}

\bonuspart[\half] Proof that if \((a,b)\) and \((c,d)\) are colinear with \( (0,0) \) , then \(ad - bc = 0\). \\

\begin{solution}
\begin{proof}
If \((a,b)\) and \((c,d)\) are colinear, the slope of the line $(a,b)$ to $(0,0)$ and the slope of the line $(c,d)$ to $(0,0)$ is equivalent. Hence,

\begin{align*}
    \frac{d}{b} &= \frac{b}{a} \\
    da &= bc \\
    ad - bc &= 0 \\
\end{align*}

\end{proof}
\end{solution}


\vspace{\stretch{1}} 

\end{parts}

\newpage



\question \textbf{Prove or disprove} (in \textbf{ZFC}) the following properties of ordinal addition. 

The definitions of ordinal addition are listed below for convinence.

$$\forall\alpha\in \textbf{Ord}$$
\begin{align*}
\textrm{(Def.\,i)}&.\,\alpha+0=\alpha\\
\textrm{(Def.\,ii)}&.\,\alpha+1=\alpha\cup\{\alpha\}\\
\textrm{(Def.\,iii)}&.\,\alpha+(\beta+1)=(\alpha+\beta)+1,\,\mathrm{for\,all}\,\beta\\
\textrm{(Def.\,iv)}&.\,\alpha+\beta=\sup\{\alpha+\xi\mid\xi<\beta\},\,\textrm{for all limit ordinal}\,\beta>0\\
\end{align*}

Note: \textbf{Ord} denotes the class of all ordinals, while $\sup$ denotes the supremum of a set.

\begin{parts}

\part[3] \(\forall\alpha,\beta,\gamma\in \textbf{Ord},\,(\alpha+\beta)+\gamma=\alpha+(\beta+\gamma)\)
\vspace{\stretch{1}}

\begin{solution}
\begin{proof}
Let $A=\{\gamma\in\textbf{Ord}\mid\forall\alpha,\beta\in\textbf{Ord},(\alpha+\beta)+\gamma=\alpha+(\beta+\gamma)\}$\\
We will prove by transfinite induction that $A=\textbf{Ord}$

0 case:
\begin{align*}
(\alpha+\beta)+0&=\alpha+\beta\\
\alpha+(\beta+0)&=\alpha+\beta\\
\end{align*}

Successor case:
\begin{align*}
(\alpha+\beta)+(\gamma+1)&=((\alpha+\beta)+\gamma)+1\\
&=(\alpha+(\beta+\gamma))+1\\
&=\alpha+(\beta+(\gamma+1))\\
\end{align*}

Limit case:
\begin{align*}
(\alpha+\beta)+\gamma&=\sup\{(\alpha+\beta)+\xi\mid\xi<\gamma\}\\
&=\sup\{\alpha+(\beta+\xi)\mid\xi<\gamma\}\\
&=\alpha+(\beta+\gamma)\\
\end{align*}
\end{proof}
\end{solution}

\vspace{\stretch{1}} 

\part[2] \(\forall\alpha,\beta\in \textbf{Ord},\,\alpha+\beta=\beta+\alpha\)

\begin{solution}
Counterexample: $\omega+1>1+\omega=\omega$
\end{solution}


\vspace{\stretch{1}} 

\end{parts}



\newpage
\end{questions}

\end{document}
