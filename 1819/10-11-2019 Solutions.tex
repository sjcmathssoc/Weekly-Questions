\documentclass[addpoints]{exam}
\usepackage{amsmath}
\usepackage{amsthm}
\usepackage{color}
\usepackage[utf8]{inputenc}

%uncomment the following line and \newpage to show solutions
\printanswers

\SolutionEmphasis{\color{red}}
\footer{}{\thepage}{}

\title{Solutions to Weekly Questions}
\author{Mathematics Society}
\date{\texttt{10-11-2019}}

\begin{document}
%This code creates the text before the first question
%-------------------------------------------------------------------


\maketitle


\textbf{Remarks:}
\begin{itemize}
    \item Note that the marks allocated \textbf{is directly proportional} to the difficulty.
    \item Bonus questions are \textbf{not necessarily more difficult}.
    \item The solution will be released on the following Wednesday.
\end{itemize}

\vspace{\stretch{.2}}


%-------------------------------------------------------------------


\begin{questions}

\question Given $\frac{1}{x^4+1} = \frac{Ax+B}{ax^2+bx+c}+\frac{Cx+D}{dx^2+ex+f}$.

\begin{parts}

\part[\half] Factorize $x^4+1$.
\begin{solution}
\begin{align*}
x^4+1&=(x^2)^2+1^2\\
&=(x^2)^2+2x^2+1^2-2x^2\\
&=(x^2+1)^2-2x^2\\
&=(x^2+1)^2-(\sqrt{2}x)^2\\
&=(x^2+1-\sqrt{2}x)(x^2+1+\sqrt{2}x)\\
&=(x^2-\sqrt{2}x+1)(x^2+\sqrt{2}x+1)
\end{align*}
\end{solution}

\vspace{\stretch{1}}

\newpage

\part[1 \half] Find solutions to \textit{a, b, c, d, e, f, A, B, C, D}.\\
Hint: partial fractions decomposition.

\begin{solution}

Let $$\frac{1}{x^4+1} = \frac{Ax+B}{x^2-\sqrt{2}x+1}+\frac{Cx+D}{x^2+\sqrt{2}x+1}$$
\begin{align*}
(x^4+1)\left(\frac{1}{x^4+1}\right) =& (x^{4}+1)\left(\frac{Ax+B}{x^2-\sqrt{2}x+1}+\frac{Cx+D}{x^2+\sqrt{2}x+1}\right)\\
1 =& (Ax+B)(x^2+\sqrt{2}x+1)\\
&(Cx+D)(x^2-\sqrt{2}x+1)\\
1 =& Ax^3+\sqrt{2}Ax^2+Ax+Bx^2+\sqrt{2}Bx+B\\
&+ Cx^3-\sqrt{2}Cx^2+Cx+Dx^2-\sqrt{2}Dx+D\\
1 =& (A+C)x^3+(\sqrt{2}A+B-\sqrt{2}C+D)x^2\\
&+ (A+\sqrt{2}B+C-\sqrt{2}D)x+(B+D)
\end{align*}
\begin{align*}
A+C &= 0\\
\sqrt{2}A+B-\sqrt{2}C+D &= 0\\
A+\sqrt{2}B+C-\sqrt{2}D &= 0\\
B+D &= 1
\end{align*}
$$A=-\frac{1}{2\sqrt{2}}, \, B=\frac{1}{2} , \, C=\frac{1}{2\sqrt{2}}, \, D=\frac{1}{2}$$
$$a=1, \, b=-\sqrt{2}, \, c=1, \, d=1, \, e=\sqrt{2}, f=1$$
\end{solution}

\vspace{\stretch{1}} 

\end{parts}

\newpage



\question Evaluate the following integrals.

\begin{parts}

\part[2] $\int\frac{x}{\sqrt{x^2+1}} \,  dx$.

\begin{solution}
Let $u =x^2+1$\\
$$dx = \frac{du}{2x}$$\\
\begin{align*}
\int\frac{x}{\sqrt{x^2+1}} \,  dx&= \int\frac{x}{\sqrt{u}} \, \frac{du}{2x}\\
&= \frac{1}{2}\int\frac{1}{\sqrt{u}} \, du\\
&= \frac{1}{2}\int u^{-\frac{1}{2}} \, du\\
&= \frac{1}{2}\cdot\frac{u^{-\frac{1}{2}+1}}{-\frac{1}{2}+1}+C\\
&= u^{\frac{1}{2}}+C\\
&= \sqrt{x^2+1}+C\\
\end{align*}
\end{solution}

\vspace{\stretch{1}}

\part[1] $\int x^\frac{x}{\ln x}\, dx$.

\begin{solution}
\begin{align*}
\int x^\frac{x}{\ln x}\, dx &= \int e^{\ln (x^\frac{x}{\ln x})}\, dx\\
&= \int e^{\frac{x}{\ln x}\ln x}\, dx\\
&= \int e^x\, dx\\
&= e^x+C
\end{align*}
\end{solution}

\vspace{\stretch{1}} 

\end{parts}

\newpage



\question Prove the following statements in \textbf{ZFC}.

\begin{parts}

\part[3] There are arbitrarily large limit ordinals. $\forall\alpha,\exists\beta>\alpha$, where $\beta$ is a limit ordinal.

\begin{solution}
\begin{proof}
Given $\alpha_{0}\in\textbf{Ord}$, define $\alpha_{n+1}=\alpha_{n}+1$. Let $\beta = \sup\{\alpha_{n}\mid n<\omega\}=\bigcup\{\alpha_{n}\mid n<\omega\}=lim_{n\to\omega}\alpha_{n}$. Since the union of ordinals is an ordinal, $\beta$ is an ordinal. And for every $\gamma<\beta$, there exists $\alpha_{n}>\gamma$, otherwise $\sup\{\alpha_{n}\mid n<\omega\}\leq\gamma$, a contradiction. Thus $\gamma+1<\alpha_{n}+1=\alpha_{n+1}<\beta$, and so $\beta$ is a limit ordinal. Therefore, there are arbitrarily large limit ordinals.
\end{proof}
\end{solution}

\vspace{\stretch{1}}

\part[3] Every normal sequence $\langle\gamma_{\alpha}\mid\alpha\in\textbf{Ord}\rangle$ has arbitrarily large fixed points, $\alpha$ such that $\gamma_{\alpha}=\alpha$.

\begin{solution}
\begin{proof}
Since $\langle\gamma_{\alpha}\mid\alpha\in\textbf{Ord}\rangle$ is increasing, for every $\beta\in\textbf{Ord}$, there exists $m\in\textbf{Ord}$ such that $\gamma_{m}>\beta$. Now let $\alpha_{0}=\gamma_{m}$, $\alpha_{n+1}=\gamma_{\alpha_{n}}$. Then $\langle\alpha_{n}\mid n\in\gamma\rangle$ is increasing. Let $\alpha=\lim_{n\to\omega}\gamma_{n}$. Repeating the argument in (a), $\alpha$ is a limit ordinal. Hence, $\alpha=\lim_{n\to\omega}\alpha_{n+1}=\lim_{n\to\omega}\gamma_{\alpha_{n}}=\lim_{\xi\to\alpha}\gamma_{\xi}=\gamma_{\lim_{\xi\to\alpha}\alpha}=\gamma_{\alpha}.$
\end{proof}
\end{solution}

\vspace{\stretch{1}}

\end{parts}

\end{questions}

\end{document}
