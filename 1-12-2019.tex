\documentclass[addpoints]{exam}
\usepackage{amsmath}
\usepackage{amsthm}
\usepackage{amssymb}
\usepackage{color}
\usepackage[utf8]{inputenc}

%uncomment the following line and \newpage to show solutions
%\printanswers

\SolutionEmphasis{\color{red}}
\footer{}{\thepage}{}

\title{Weekly Questions}
\author{Mathematics Society}
\date{\texttt{1-12-2019}}

\begin{document}
%This code creates the text before the first question
%-------------------------------------------------------------------


\maketitle


\textbf{Remarks:}
\begin{itemize}
    \item Note that the marks allocated \textbf{is directly proportional} to the difficulty.
    \item Bonus questions are \textbf{not necessarily more difficult}.
    \item The solution will be released on the following Wednesday.
\end{itemize}

\vspace{\stretch{.2}}


%-------------------------------------------------------------------


\begin{questions}
\question Prove by mathematical induction the following:

\begin{parts}

\part[1] $1+2+3+…+n=\frac{n(n+1)}{2}$.
\begin{solution}
\begin{proof}
The proof will be spilt into two parts. The first part is showing that the formula holds for $1$, and the second part is showing that if the formula holds for any $k\in\mathbb{N}$, then it holds for $k+1$ as well.\\
We show that the formula holds for 1:\\
\begin{align*}
1&=\frac{1(1+1)}{2}\\
&=1\\
\end{align*}
Then, we will prove the successor case:\\
\begin{align*}
1+2+3+…+k+(k+1)&=\frac{k(k+1)}{2}+(k+1)\\
&=\frac{(k+1)(k+1)}{2}\\
&=\frac{(k+1)((k+1)+1)}{2}\\
\end{align*}
Thus, we have shown that the formula holds for all natural numbers.
\end{proof}
\end{solution}
\vspace{\stretch{1}}
%\newpage

\part[1] $1^{2}+2^{2}+3^{2}+…+n^{2}=\frac{n(n+1)(2n+1)}{6}$.

\begin{solution}
\begin{proof}
We will prove this also by induction.\\
\begin{align*}
1^{2}&=\frac{1(1+1)(2+1)}{6}\\
&=1\\
\end{align*}
Successor case:
\begin{align*}
1^{2}+2^{2}+3^{2}+…+k^{2}+(k+1)^{2}&=\frac{k(k+1)(2k+1)}{6}+(k+1)^{2}\\
&=\frac{k(k+1)(2k^{2}+k+6k+1)}{6}\\
&=\frac{(k+1)((k+1)+1)(2(k+1)+1)}{6}\\
\end{align*}
\end{proof}
\end{solution}
\end{parts}
\vspace{\stretch{1}}
%\newpage



\question[2] Let $(G,\circ)$ be a group and let $(S_{1},\circ)$ and $(S_{2},\circ)$ be it's subgroups. Prove that $(S_{1}\cap S_{2},\circ)$ is a subgroup of $(G,\circ)$.\\

The conditions for being a subgroup (and a group) are listed below for convenience.\\
Let $(G,\circ)$ be a group. Assume $S$ is a subset of $G$ and $(S,\circ)$ is a group. Then $(S,\circ)$ is a subgroup of $(G,\circ)$.\\
The requirements for being a group are listed below. $(H,\circ)$ is a group (where $\circ$ is a binary operation on $H$) if and only if:

\begin{align*}
\textrm{(i)}&.\,\forall a,b\in H, a\circ b\in H.\\
\textrm{(ii)}&.\,\forall a,b,c\in H, (a\circ b)\circ c=a\circ(b\circ c).\\
\textrm{(iii)}&.\,\exists e\in H\,\forall a\in H, a\circ e=e=e\circ a.\\
\textrm{(iv)}&.\,\forall a\in H\,\exists a', a\circ a'=a'\circ a=e.\\
\end{align*}

\begin{solution}
\begin{proof}
We will show that $(S_{1}\cap S_{2},\circ)$ satisfies the conditions for being a subgroup of $(G,\circ)$.\\
\\
$S_{1}\cap S_{2}$ is obviously a subset of $G$.
\\
For every $a,b\in S_{1}\cap S_{2}$, $a,b\in S_{1}$ and $a,b\in S_{2}$, thus by definition $a\circ b\in S_{1}$ and $a\circ b\in S_{2}$, therefore $a\circ b\in S_{1}\cap S_{2}$.\\
For every $a,b,c\in S_{1}\cap S_{2}$, $a,b,c\in S_{1}$ and $a,b,c\in S_{2}$, thus by definition $(a\circ b)\circ c=a\circ(b\circ c)$.\\
$e\in S_{1}$ and $e\in S_{2}$, therefore $e\in S_{1}\cap S_{2}$.\\
For every $a\in S_{1}$, $a'\in S_{1}$ and similarly for any $a\in S_{2}$, $a'\in S_{2}$. Thus for any $a\in S_{1}\cap S_{2}$, $a'\in S_{1}\cap S_{2}$.
\end{proof}
\end{solution}
\vspace{\stretch{1}}
%\newpage



\question[3] Prove that the set of all real numbers $\mathbb{R}$ is uncountable.

\begin{solution}
\begin{proof}
Assume that $\mathbb{R}$ is countable, and let $j_{0}$, $j_{1}$, $j_{2}$, …,$j_{n}$, …$n\in \mathbb{N}$ be an enumeration of $\mathbb{R}$. Let $a_{0}=j_{0}$ and $y_{0}=j_{s_{0}}$, where $s_{0}$ is the least $s$ such that $a_{0}<j_{s}$. For each $n$ define $a_{n+1}=j_{q_{n}}$ where $q_{n}$ is the least $q$ such that $a_{n}<j_{q}<y_{n}$, and define $y_{n+1}=j_{s_{n}}$, where $s_{n}$ is the least $s$ such that $a_{n+1}<j_{s}<y_{n}$. Now let $a=\sup\{a_{n}\mid n\in\mathbb{N}\}$, it is obvious that $a\neq j_{m}$ for all $m$. Thus $\mathbb{R}$ is uncountable.
\end{proof}
\end{solution}
\vspace{\stretch{1}}



\end{questions}

\end{document}