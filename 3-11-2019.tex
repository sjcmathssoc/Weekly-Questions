\documentclass[addpoints]{exam}
\usepackage{amsmath}
\usepackage[utf8]{inputenc}
\footer{}{}{}

\title{Weekly Questions}
\author{Mathematics Society}
\date{\texttt{3-11-2019}}

\begin{document}
%This code creates the text before the first question
%-------------------------------------------------------------------


\maketitle



\textbf{Remarks:}
\begin{itemize}
    \item Note that the marks allocated \textbf{is directly proportional} to the difficulty.
    \item Bonus questions are \textbf{not necessarily more difficult}.
    \item The solution will be released on the following Wednesday.
\end{itemize}

\vspace{\stretch{.2}}




%-------------------------------------------------------------------


\begin{questions}

\question Given \(ax^2 + bx + c = 0\) and \(b^2 > 4ac\), derive the quadratic equation \(x=\frac{-b\pm\sqrt{b^{2}-4ac}}{2a}\).

\begin{parts}

\part[2 \half] Find \(e, f, g\) expressed by \(a, b, c\) such that \(ax^2 + bx + c = e(x+f)^2 + g\). \\
Hint: \( g < 0 \)

\vspace{\stretch{1}} 

\part[\half] Factorize the solution for part a and hence derive the quadratic equation.

\vspace{\stretch{1}} 

\end{parts}



\question Given two coordinates \( (a,b)\) and \((c, d)\).

\begin{parts}

\part[1 \half] Show that if \(ad - bc \neq 0\), then there \textbf{does not exist} \(e,f \neq 0  \) such that \( (ea, eb) + (fc, fd) = (0, 0) \). \\ 
\vspace{\stretch{1}} 

\part[1 \half] Show that if \(ad - bc = 0\), then there \textbf{exists} \(e,f \neq 0  \) such that \( (ea, eb) + (fc,fd) = (0, 0) \). \\ 
\vspace{\stretch{1}}

\bonuspart[\half]  Proof that if \((a,b)\) and \((c,d)\) are colinear with \( (0,0) \) , then \(ad - bc = 0\). \\
\vspace{\stretch{1}}

\end{parts}



\question \textbf{Prove or disprove} (in \textbf{ZFC}) the following properties of ordinal addition. 

The definitions of ordinal addition are listed below for convinence.

$$\forall\alpha\in \textbf{Ord}$$
\begin{align*}
\textrm{(Def.\,i)}&.\,\alpha+0=\alpha\\
\textrm{(Def.\,ii)}&.\,\alpha+1=\alpha\cup\{\alpha\}\\
\textrm{(Def.\,iii)}&.\,\alpha+(\beta+1)=(\alpha+\beta)+1,\,\mathrm{for\,all}\,\beta\\
\textrm{(Def.\,iv)}&.\,\alpha+\beta=\sup\{\alpha+\xi\mid\xi<\beta\},\,\textrm{for all limit ordinal}\,\beta>0\\
\end{align*}

Note: \textbf{Ord} denotes the class of all ordinals, while $\sup$ denotes the supremum of a set.

\begin{parts}

\part[3] \( \forall\alpha,\beta,\gamma\in \textbf{Ord},\,(\alpha+\beta)+\gamma=\alpha+(\beta+\gamma)\)
\vspace{\stretch{1}}

\part[2] \(\forall\alpha,\beta\in \textbf{Ord},\,\alpha+\beta=\beta+\alpha\)
\vspace{\stretch{1}}


\end{parts}

\end{questions}

\end{document}
